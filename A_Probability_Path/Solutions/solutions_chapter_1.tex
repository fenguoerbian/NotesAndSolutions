\section{Solutions to Chapter 1: Sets and Events}
\label{sec:solutions-chapter-1}

\newcounter{Lcount}
\setcounter{Lcount}{0}
\begin{list}{1.9.\arabic{Lcount}}{\usecounter{Lcount}}
\item \label{ex.1.9.1}
  $\forall B \in \aleph$, since $\mathcal{C}\subset B$, we have $\{0\}\in B$, therefore $\Omega\setminus \{0\}=\{1\}\in B$. Also $\emptyset\in B$ and $\Omega\in B$. Therefore $\{\emptyset,\{0\}, \{1\}, \Omega\}\subset B$. Note that $\mathcal{P}\left(\Omega\right) = \{\emptyset,\{0\}, \{1\}, \Omega\}$. This means
  \[
    \aleph = \{\mathcal{P}\left(\Omega\right)\}
  \]
\item \label{ex.1.9.2}Like in 1.9.\ref{ex.1.9.1}, we can conclude that
  \[
    \forall B \in \aleph \quad \Rightarrow \left\{\emptyset, \{0\}, \{1,2\}, \Omega\right\} \subset B
  \]
  Also note that $\left\{\emptyset, \{0\}, \{1,2\}, \Omega\right\}$ is a $\sigma$-field itself which means
  \[
    \sigma\left(\mathcal{C}\right) = \left\{\emptyset, \{0\}, \{1,2\}, \Omega\right\}
  \]
  Those subsets of $\Omega$ which are not included in $\sigma(\mathcal{C})$ are
  \[
    \{1\},\quad \{2\},\quad \{0,1\},\quad \{0,2\}
  \]
  and it's easy to check that they are all included in $B$ if any one of them is inclued. So to sum up, we have
  \[
    \aleph = \left\{\sigma(\mathcal{C}), \mathcal{P}\left(\Omega\right)\right\}
  \]
\item \label{ex.1.9.3} Firstly
  \[
    \begin{aligned}
      \underset{n\to\infty}{\mathrm{lim\,sup}} A_n\cup B_n
      &= \left\{x\middle| \sum\limits_{n=1}^\infty 1_{A_n\cup B_n}\left(x\right)=\infty\right\}    \\
      &= \left\{x\middle| \sum\limits_{n=1}^\infty 1_{A_n}\left(x\right)=\infty\quad \mathrm{or}\quad
         \sum\limits_{n=1}^\infty 1_{B_n}\left(x\right)=\infty\right\}    \\
      &= \left\{x\middle| \sum\limits_{n=1}^\infty 1_{A_n}\left(x\right)=\infty \right\} \cup
         \left\{x\middle| \sum\limits_{n=1}^\infty 1_{B_n}\left(x\right)=\infty\right\}    \\
      &= \underset{n\to\infty}{\mathrm{lim\,sup}} A_n \cup \underset{n\to\infty}{\mathrm{lim\,sup}} B_n
    \end{aligned}
  \]
  Secondly, the statement
  \[
    A_n\cup B_n \to A\cup B,\quad A_n\cap B_n \to A \cap B
  \]
  is true if $A_n\to A$ and $B_n \to B$. Because we have
  \[
    \begin{aligned}
      & \underset{n\to\infty}{\mathrm{lim\,sup}} A_n
      = \underset{n\to\infty}{\mathrm{lim\,inf}} A_n
      = \underset{n\to\infty}{\mathrm{lim}} A_n = A    \\
      & \underset{n\to\infty}{\mathrm{lim\,sup}} B_n
      = \underset{n\to\infty}{\mathrm{lim\,inf}} B_n
      = \underset{n\to\infty}{\mathrm{lim}} B_n = B
    \end{aligned}
  \]
  Using the result of the first problem we can deduce that
  \[
    \underset{n\to\infty}{\mathrm{lim\,sup}} A_n\cup B_n =
    \underset{n\to\infty}{\mathrm{lim\,sup}} A_n \cup \underset{n\to\infty}{\mathrm{lim\,sup}} B_n
    = A \cup B
  \]
  We now have to show that
  \[
    \underset{n\to\infty}{\mathrm{lim\,inf}} A_n \cup B_n =
    \underset{n\to\infty}{\mathrm{lim\,inf}} A_n \cup \underset{n\to\infty}{\mathrm{lim\,inf}} B_n
    = A \cup B
  \]
  Or equally
  \[
    \underset{n\to\infty}{\mathrm{lim\,sup}} A_n \cup B_n
    \subset
    \underset{n\to\infty}{\mathrm{lim\,inf}} A_n \cup B_n
  \]
  \[
    \begin{aligned}
      & x \in \underset{n\to\infty}{\mathrm{lim\,sup}} A_n \cup B_n
      \iff x \in A \cup B
      \iff \underset{n\to\infty}{\mathrm{lim\,inf}} A_n \cup \underset{n\to\infty}{\mathrm{lim\,inf}} B_n\\
      \iff&
      \left\{x \notin A_n\text{, finitely}\right\} \text{ or } \left\{x \notin B_n\text{, finitely}\right\}    \\
      \implies& \left\{ x \notin A_n \cup B_n \text{, finitely}\right\}
      \iff x \in \underset{n\to\infty}{\mathrm{lim\,inf}} A_n \cup B_n
    \end{aligned}
  \]
  This means $\forall x\in \underset{n\to\infty}{\mathrm{lim\,sup}} A_n \cup B_n$, we have that $x\in\underset{n\to\infty}{\mathrm{lim\,inf}} A_n \cup B_n$, therefore
  \[
    \underset{n\to\infty}{\mathrm{lim\,sup}} A_n \cup B_n
    \subset
    \underset{n\to\infty}{\mathrm{lim\,inf}} A_n \cup B_n
  \]
  which means
  \[
    A_n \cup B_n \to A \cup B
  \]
  and
  \[
    A_n \cap B_n = \left(A_n^c \cup B_n^c\right)^c \to \left(A^c \cup B_c\right)^c = A \cap B
  \]
\item \label{ex.1.9.4}
  \[
    \begin{aligned}
      \underset{n\to\infty}{\mathrm{lim\,inf}} A_n &= \bigcup\limits_{n=1}^\infty \bigcap\limits_{k=n}^\infty A_k    \\
      &= \bigcup\limits_{n=1}^\infty \bigcap\limits_{k=n}^\infty \left\{
        \frac{m}{k}:m\in\mathbb{N}
      \right\}    \\
      &= \bigcup\limits_{n=1}^\infty \mathbb{N} = \mathbb{N}
    \end{aligned}
  \]
  \[
    \begin{aligned}
      \underset{n\to\infty}{\mathrm{lim\,sup}} A_n &= \bigcap\limits_{n=1}^\infty \bigcup\limits_{k=n}^\infty A_k    \\
      &= \bigcap\limits_{n=1}^\infty \bigcup\limits_{k=n}^\infty \left\{
        \frac{m}{k}:m\in\mathbb{N}
      \right\}    \\
      &= \bigcap\limits_{n=1}^\infty\mathbb{Q^+} = \mathbb{Q^+}
    \end{aligned}
  \]
  
\item
  \[
    \begin{aligned}
      & \left\{\omega:f_n\left(\omega\right)\nrightarrow f\left(\omega\right)\right\}    \\
      \iff & \left\{\omega: \exists\;\epsilon > 0,\;\mathrm{s.t.}\;\forall N,\;\exists\;n>N,\;\mathrm{s.t.}\;
        \left|f_n\left(\omega\right) - f\left(\omega\right)\right|>\epsilon
      \right\}    \\
      \iff & \bigcup\limits_{k=1}^\infty\bigcap\limits_{N=1}^\infty\bigcup\limits_{n=N}^\infty
      \left\{
        \omega:\left|f_n\left(\omega\right) - f\left(\omega\right)\right| > \frac{1}{k}
      \right\}
    \end{aligned}
  \]
  
\item
  Use Lemma 1.3.1, we can conclude that
  \[
    \underset{n\to\infty}{\mathrm{lim\,sup}}\,A_n = \underset{n\to\infty}{\mathrm{lim\,inf}}\,A_n
    = \left(0,\;1\right]
  \]
\item 
  \begin{enumerate}
  \item Since $\theta = 1/8 $, the period is $T=8$. And there are actually 2 distinguished squares. Hence $ \underset{n\to\infty}{\mathrm{limsup}}\;I_n$ is the star area covered by at least one squate and $ \underset{n\to\infty}{\mathrm{liminf}}\;I_n$ is the area covered by both squares. Refer to Figure~\ref{fig:1.9.7.a} as illustration.
    \begin{figure}[htbp]
      \centering
      \includegraphics[width=0.6\textwidth]{./Figures/1_9_7_a.pdf}
      \caption{(a)}
      \label{fig:1.9.7.a}
    \end{figure}

  \item If $\theta$ is rational, then it can be written in the form $\theta = \frac{m}{n}$ where both $m$ and $n$ are integers, which means there is a period in $I_n$. Hence like before, $ \underset{n\to\infty}{\mathrm{limsup}}\;I_n$ is the star area covered by at least one squate and $ \underset{n\to\infty}{\mathrm{liminf}}\;I_n$ is the area covered by all squares. Refer to Figure~\ref{fig:1.9.7.b} as illustration.
    \begin{figure}[htbp]
      \centering
      \includegraphics[width=0.6\textwidth]{./Figures/1_9_7_b.pdf}
      \caption{(b) $\theta = \frac{1}{7}$}
      \label{fig:1.9.7.b}
    \end{figure}
    
  \item If $\theta$ is irrational. These squares becomes dense and $ \underset{n\to\infty}{\mathrm{limsup}}\;I_n$ is the round area with radius $r_{\mathrm{sup}}= \sqrt{2}$ and $ \underset{n\to\infty}{\mathrm{liminf}}\;I_n$ is the round area with radius $r_{\mathrm{inf}} = 1$. Refer to Figure~ as illustration.
    \begin{figure}[htbp]
      \centering
      \includegraphics[width=0.6\textwidth]{./Figures/1_9_7_c.pdf}
      \caption{(c) $\theta = e^{1/2}$}
      \label{fig:1.9.7.c}
    \end{figure}
  \item Codes for drawing these figures are provided below:
    \lstinputlisting[language=R]{./Codes/ex_1_9_7_plot.r}
  \end{enumerate}

  
\item To be added.

  
\end{list}

%%% Local Variables:
%%% mode: latex
%%% TeX-master: "../Solutions_A_Probability_Path"
%%% End:
