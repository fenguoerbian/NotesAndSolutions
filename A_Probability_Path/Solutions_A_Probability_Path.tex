\documentclass[a4paper,12pt]{article}
\usepackage{geometry}
\geometry{left=2.5cm,right=2.5cm,top=2.5cm,bottom=2.5cm}
\renewcommand{\textfraction}{0.15}
\renewcommand{\topfraction}{0.85}
\renewcommand{\bottomfraction}{0.65}
\renewcommand{\floatpagefraction}{0.60}
\usepackage{amsmath}
\usepackage{amsfonts}
\usepackage{mathrsfs}
\usepackage{amsthm}
\usepackage{extarrows}
\usepackage{bm}
\usepackage{graphicx}
\usepackage[section]{placeins}
\usepackage{flafter}
\usepackage{array}
\usepackage{caption}
\usepackage{subcaption}
\usepackage{color}
\usepackage{multirow}


\DeclareMathOperator*{\argmaxdown}{arg\,max}
\DeclareMathOperator*{\argmindown}{arg\,min}
\DeclareMathOperator{\argmax}{arg\,max}
\DeclareMathOperator{\argmin}{arg\,min}


\title{Solutions of A Probabilit Path}
\author{Chao Cheng
  \\
  Github ID: fenguoerbian
  \\
  Mail: 413557584@qq.com}
\date{\today}



\begin{document}
\maketitle

\section{Solutions to Chapter 1: Sets and Events}
\label{sec:solutions-chapter-1}

\newcounter{Lcount}
\setcounter{Lcount}{0}
\begin{list}{1.9.\arabic{Lcount}}{\usecounter{Lcount}}
\item \label{ex.1.9.1}
  $\forall B \in \aleph$, since $\mathcal{C}\subset B$, we have $\{0\}\in B$, therefore $\Omega\setminus \{0\}=\{1\}\in B$. Also $\emptyset\in B$ and $\Omega\in B$. Therefore $\{\emptyset,\{0\}, \{1\}, \Omega\}\subset B$. Note that $\mathcal{P}\left(\Omega\right) = \{\emptyset,\{0\}, \{1\}, \Omega\}$. This means
  \[
    \aleph = \{\mathcal{P}\left(\Omega\right)\}
  \]
\item \label{ex.1.9.2}Like in 1.9.\ref{ex.1.9.1}, we can conclude that
  \[
    \forall B \in \aleph \quad \Rightarrow \left\{\emptyset, \{0\}, \{1,2\}, \Omega\right\} \subset B
  \]
  Also note that $\left\{\emptyset, \{0\}, \{1,2\}, \Omega\right\}$ is a $\sigma$-field itself which means
  \[
    \sigma\left(\mathcal{C}\right) = \left\{\emptyset, \{0\}, \{1,2\}, \Omega\right\}
  \]
  Those subsets of $\Omega$ which are not inclued in $\sigma(\mathcal{C})$ are
  \[
    \{1\},\quad \{2\},\quad \{0,1\},\quad \{0,2\}
  \]
  and it's easy to check that they are all inclued in $B$ if any one of them is inclued. So to sum up, we have
  \[
    \aleph = \left\{\sigma(\mathcal{C}), \mathcal{P}\left(\Omega\right)\right\}
  \]
\item \label{ex.1.9.3} 
\end{list}











\clearpage
\appendix

\end{document}




%%% Local Variables:
%%% mode: latex
%%% TeX-master: t
%%% End:
