\documentclass[a4paper,12pt]{article}
\usepackage{geometry}
\geometry{left=2.5cm,right=2.5cm,top=2.5cm,bottom=2.5cm}
\renewcommand{\textfraction}{0.15}
\renewcommand{\topfraction}{0.85}
\renewcommand{\bottomfraction}{0.65}
\renewcommand{\floatpagefraction}{0.60}
\usepackage{amsmath}
\usepackage{amsfonts}
\usepackage{mathrsfs}
\usepackage{amsthm}
\usepackage{extarrows}
\usepackage{bm}
\usepackage{graphicx}
\usepackage[section]{placeins}
\usepackage{flafter}
\usepackage{array}
\usepackage{caption}
\usepackage{subcaption}
\usepackage{color}
\usepackage{multirow}


\DeclareMathOperator*{\argmaxdown}{arg\,max}
\DeclareMathOperator*{\argmindown}{arg\,min}
\DeclareMathOperator{\argmax}{arg\,max}
\DeclareMathOperator{\argmin}{arg\,min}


\title{Notes of A Probabilit Path}
\author{Chao Cheng\quad 413557584@qq.com}
\date{\today}



\begin{document}
\maketitle

This is my personal notes while reading the book A Probability Path written by Sidney I. Resnick. It's not a complete summary of all the concepts and/or informations in the book. Just a reminder to myself about the new knowledge I got when I read the book or something I was confused before. 

\section{Sets and Events}
\label{sec:sets-events}









\clearpage
\appendix

\end{document}




%%% Local Variables:
%%% mode: latex
%%% TeX-master: t
%%% End:
